В~\nameref{sec:extras} представлена программа на языке Python 3.12\cite{python},
реализованная в среде разработки PyCharm Community Edition 2024.3.1,
численного решения одномерного стационарного уравнения Шрёдингера для электрона в одномерной потенциальной яме.
Программа реализует алгоритм теории возмущений,
позволяющий находить собственные значения и соответствующие им волновые функции.
Потенциальная функция (невозмущенная система) и параметры для нее соответствуют постановке задачи из первой главы.
Энергия и длина ямы были переведены в атомные единицы Хартри (строки 220--224).


В строках 13--14 определена потенциальная функция.


В строках 17--23 реализована функция вычисляющая базисную волновую функцию $k$-го состояния.


В строках 48--51 реализована функция, вычисляющая матричный элемент по формулам (\ref{eq:gamilton_op_components}, ~\ref{eq:gamilton_kinetic_e}, ~\ref{eq:gamilton_pot_e}), функция реализованная в строках~\textbf{Х}--\textbf{Х} является вспомогательной и вычисляет вторую производную для заданной функции.


В строках 54--59 реализовано построение матрицы Гамильтона.


В строках 62--64 реализована функция вычисляющая собственные значения и собственные вектора заданной матрицы.


В строках 67--71 реализована функция вычисляющая волновую функцию по формуле~\eqref{eq:ortonorm_basis_func}


В строках 74--87 реализованы функции вычисляющие квантовомеханические средние $\langle p(x) \rangle$ и $\langle p(x^2) \rangle$.


В строках 90--129 реализована функция выводящая графики волновых функций.

В строках 138--217 реализован целевой метод - метод пристрелки,
разобранный в первой лабораторной работе,
с которым будет сравниваться решение полученное текущим методом.

В строках 227--228 задаются размерность сетки и матрицы Гамильтона.


В строках 242--258 вычисляются энергии и волновые функции и производится запись результата вычислений в файл.