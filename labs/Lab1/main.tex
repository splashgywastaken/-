\documentclass[a4paper,12pt]{article}


\usepackage{mathtext}
\usepackage[T2A]{fontenc}
\usepackage[utf8]{inputenc}
\usepackage[russian]{babel}
\usepackage{amsmath}
\usepackage{amsfonts}
\usepackage{amssymb}
\usepackage{graphicx}
\usepackage{nameref}
\usepackage{subcaption}
\usepackage[margin=1.5pt]{geometry}
\usepackage{float}
\usepackage{latexsym}
\usepackage{stmaryrd}
\usepackage{hyperref}
\usepackage{listings}
\usepackage{xcolor}
\usepackage{tabularx}


\geometry{left=2cm, right=2cm, top=2cm, bottom=2cm}

\begin{document}

\definecolor{codegreen}{rgb}{0,0.6,0}
\definecolor{codegray}{rgb}{0.5,0.5,0.5}
\definecolor{codepurple}{rgb}{0.58,0,0.82}
\definecolor{backcolour}{rgb}{1,1,1}

\lstdefinestyle{pythonStyle}{
    backgroundcolor=\color{backcolour},
    commentstyle=\color{codegreen},
    keywordstyle=\color{red},
    numberstyle=\tiny\color{codegray},
    stringstyle=\color{codepurple},
    basicstyle=\ttfamily\footnotesize,
    breakatwhitespace=false,
    breaklines=true,
    captionpos=b,
    keepspaces=true,
    numbers=left,
    numbersep=5pt,
    showspaces=false,
    showstringspaces=false,
    showtabs=false,
    tabsize=2
}
\lstset{style=pythonStyle}

\pagestyle{empty}

\begin{center}
\textbf{МИНОБРНАУКИ РОССИИ}\\
ФЕДЕРАЛЬНОЕ ГОСУДАРСТВЕННОЕ БЮДЖЕТНОЕ \\
ОБРАЗОВАТЕЛЬНОЕ УЧРЕЖДЕНИЕ ВЫСШЕГО ОБРАЗОВАНИЯ \\
ВОРОНЕЖСКИЙ ГОСУДАРСТВЕННЫЙ УНИВЕРСИТЕТ \\
Факультет прикладной математики, информатики и механики\\
Кафедра вычислительной математики и прикладных информационных технологий
\end{center}

\vspace{2cm}
\begin{center}
\textbf{ЛАБОРАТОРНАЯ РАБОТА №1}\\
\textbf{ЧИСЛЕННОЕ РЕШЕНИЕ СТАЦИОНАРНОГО УРАВНЕНИЯ ШРЁДИНГЕРА: МЕТОД ПРИСТРЕЛКИ}
\end{center}

\vspace{3cm}
\begin{flushright}
\begin{tabular}{l l}
\textbf{Направление:} & 01.04.02 \textendash{} Прикладная математика и информатика \\
\textbf{Выполнил:} & студент 11 группы 2 курса магистратуры \\
& Крутько А.С. \\
\textbf{Преподаватель:} & доктор физ.-мат. наук, профессор Тимошенко Ю.К.
\end{tabular}
\end{flushright}

\vspace{3cm}
\begin{center}
Воронеж 2024
\end{center}

\newpage
\tableofcontents
\pagestyle{plain}
\setcounter{page}{2}

\newpage
\section{Цели и задачи работы}\label{sec:goals}
\subsection{Цель работы.}\label{subsec:-.}
Целями лабораторной работы являются практическое освоение информации, полученной при изучении курса <<Компьютерное моделирование в математической физике>> по теме <<Численное решение стационарного уравнения Шрёдингера>>, а также развитие алгоритмического мышления и приобретение опыта использования знаний и навыков по математике, численным методам и программированию для решения прикладных задач физико-технического характера.

\subsection{Задачи работы:}\label{subsec:-:}

\textbf{Проблема:} электрон находится в одномерной потенциальной яме с бесконечными стенками $U(x)$:
\[
v(x) =
    \begin{cases}
        J_2(x), & x \in (-L, L), \\
        \infty, & x \notin (-L, L),
    \end{cases}
\]
Где $U(x) = v(x)*V_0$, $V_0 = 25$ \text{эВ}, $L = 3$ \AA, $J_n(x)$ -- функция Бесселя, $n = 2$.

\begin{enumerate}
    \item Используя метод пристрелки, найти энергии, нормированные волновые функции и плотности вероятности для основного и 2-го возбужденного состояний.
          Привести как численные значения энергий, так и построить графики волновых функций и плотностей вероятности.
    \item Вычислить для этих состояний квантовомеханические средние $\langle p(x) \rangle$ и $ \langle p(x^2) \rangle $.
\end{enumerate}

\newpage
\section{Одномерное стационарное уравнение Шрёдингера. Математический формализм. Общие свойства решений}\label{sec:-}
Одномерное стационарное уравнение Шрёдингера~\cite{tim}:
\begin{equation}
    \hat{H}\psi(x) = E\psi(x),
    \label{eq:oneDimShrodingerEq}
\end{equation}
где $\hat{H}$ \textendash{} оператор Гамильтона, $E$ \textendash{} собственные значения энергии, $\psi(x)$ \textendash{} волновая функция.

С математической точки зрения оно представляет собой задачу определения собственных значений $E$ и собственных функций $\psi$ оператора Гамильтона $\hat{H}$.
Для частицы с массой $m$, находящейся в потенциальном поле $U(x)$, оператор Гамильтона имеет вид
\begin{equation}
    \hat{H} = \hat{T}+ U(x),
    \label{eq:equation}
\end{equation}
где оператор кинетической энергии
\begin{equation}
    \hat{T} =-\frac{\hslash^2}{2m}\frac{d^2}{dx^2},
    \label{eq:equation2}
\end{equation}
а $\hslash$ --- постоянная Планка.
Собственное значение оператора Гамильтона имеет смысл энергии соответствующей изолированной квантовой системы.
Собственные функции называются волновыми функциями.
Волновая функция однозначна и непрерывна во всём пространстве.
Непрерывность волновой функции и её первой производной сохраняется и при обращении $U(x)$ в $\infty$ некоторой области пространства.
В такую область частица вообще не может проникнуть, то есть в этой области, а также на её границе $\psi(x)=0$.

Оценим нижнюю границу энергетического спектра.
Пусть минимальное значение потенциальной функции равно $U_{\min}$.
Очевидно, что $\langle T \rangle \geq 0$ и $\langle U \rangle \geq U_{\min}$.
Потому из уравнения~\eqref{eq:oneDimShrodingerEq} следует, что:
\begin{equation}
    E =\langle H\rangle = \int_{-\infty}^{+\infty} \psi^\star(x)\hat{H}\psi(x) \,dx =\langle T\rangle +\langle U\rangle > U_{\min}.
\label{eq:e_h_integral}
\end{equation}
то есть, энергии всех состояний > $U_{min}$.

Особый практический интерес представляет случай, когда
\begin{equation}
    \lim_{x\to\infty} U(x) = 0.
\label{eq:limit_pot_inf}
\end{equation}

Потенциал такого типа называется также потенциальной ямой.
Для данной $U(x)$ свойства решений уравнения Шрёдингера зависят от знака собственного значения $E$.
Если $E < 0$.
Частица с отрицательной энергией совершает финитное движение.
Оператор Гамильтона имеет дискретный спектр, то есть собственные значения и соответствующие собственные функции можно снабдить номерами.
При~$E < 0$ уравнение~\eqref{eq:oneDimShrodingerEq} приобретает вид\cite{tim}:

\begin{equation}
    \hat{H}\psi_k(x) = E_k\psi_k(x).
    \label{eq:shrodinger_eq_e_less_0}
\end{equation}

Квантовое состояние, обладающее наименьшей энергией, называется основным.
Остальные состояния называют возбужденными состояниями.
В силу линейности стационарного уравнения Шрёдингера, волновые функции математически определены с точностью до постоянного множителя.
Однако, из физических соображений, волновые функции должны быть нормированы следующим образом:

\begin{equation}
    \int_{-\infty}^{+\infty} |\psi_k(x)|^2, dx = 1.
    \label{eq:wave_func_normalization}
\end{equation}

В дальнейшем будет рассматриваться только дискретный спектр.
При этом необходимо пользоваться \textbf{осцилляционной теоремой}.

\textbf{Осцилляционная теорема.}
Упорядочим собственные значения оператора Гамильтона в порядке возрастания, нумеруя энергию основного состояния индексом "0": $E_0$, $E_1$, $E_2$ \dots, $E_k$,\dots.
Тогда волновая функция $\psi_k(x)$ будет иметь $k$ узлов (то есть, пересечений с осью абсцисс).
Исключения: области, в которых потенциальная функция бесконечна.

\section{Метод пристрелки. Алгоритм}\label{sec:shooting_method}

Прежде чем перейти к алгоритму пристрелки нужно преобразовать уравнение~\eqref{eq:oneDimShrodingerEq}.
В данном случае граничные условия для волновой функции~\cite{tim}:

\begin{equation}
    \psi(a) =\psi(b) = 0
    \label{eq:waveFuncBordCond}
\end{equation}
где в точках a и b по оси абсцисс построены бесконечные потенциальные стенки.
Для решения уравнения Шрёдингера удобно использовать атомные единицы Хартри (\(e = 1, \hslash = 1\) \text{и} $m_e = 1$).
В этих единицах уравнение~\eqref{eq:shrodinger_eq_e_less_0}, предполагая, что $m=m_e$~приобретает вид~\cite{tim}:

\begin{equation}
    \left[ -\frac{1}{2}\frac{d^2}{dx^2} +U(x) \right]\psi{x}=E\psi{x}.
    \label{eq:shrod_au}
\end{equation}

Преобразуем~\eqref{eq:shrod_au} к форме:

\begin{equation}
    \frac{d^2\psi{x}}{dx^2} + q(E, x)\psi(x)=0,
    \label{eq:shrod_au_converted_form}
\end{equation}
где
\begin{equation}
    q(E, x)=2\left[ E - U(x) \right].
    \label{eq:q_e_x}
\end{equation}

Решение стационарного уравнения Шрёдингера сводится к нахождению собственных значений и собственных функций оператора Гамильтона, так как для собственных значений известна оценка снизу~\eqref{eq:e_h_integral}, то удобно начинать с вычисления энергии и волновой функции основного состояния.
Оценим грубо энергию основного состояния $E_0^{(0)}=U_{min} + \delta,$где $\delta$ - малая величина ($\delta > 0$).
Подставим значение этой энергии в уравнение~\eqref{eq:shrod_au}.
Это уравнение теперь становится обыкновенным дифференциальным уравнением 2-го порядка с граничными условиями~\eqref{eq:waveFuncBordCond}.
Рассмотрим алгоритм, использующий эту идею.

Зададим на интервале $\left[ a,b \right]$ сетку из $N$ узлов с постоянным шагом $h = \frac{(b-a)}{N-1}$:

\begin{equation}
    x_n=a+(n-1)h, n = 1,2,3,\dots,N.
    \label{eq:x_nodes}
\end{equation}

Граничные условия~\eqref{eq:waveFuncBordCond} приобретают вид:

\begin{equation}
    \psi_1 = \psi_N = 0.
    \label{eq:newWaveFuncBordCond}
\end{equation}

Задача Коши для дифференциального уравнения~\eqref{eq:shrod_au} часто решается методом Нумерова.
В рамках этого метода значения функции в узле сетки находят интегрируя \guillemetleftвперёд\guillemetright:

\begin{equation}
    \psi_{n+1} = \left[ 2(1 - 5cq_n)\psi_n - (1 + cq_{n-1})\psi_{n-1} \right]\left( 1 + cq_{n+1}^{-1} \right),
    \label{eq:integrForward}
\end{equation}
либо интегрируя \guillemetleftназад\guillemetright
\begin{equation}
    \psi_{n+1} = \left[ 2(1 - 5cq_n)\psi_n - (1 + cq_{n+1})\psi_{n+1} \right]\left( 1 + cq_{n-1}^{-1} \right),
    \label{eq:integrBackward}
\end{equation}

Здесь $c=h^2/12, q_n=q(E, x_n)$.
При использовании формулы~\eqref{eq:integrForward} необходимо знать $\psi_1$ и $\psi_2$, а формулы~\eqref{eq:integrBackward} --- $\psi_{N-1}$ и $\psi_N$.
Значения $\psi_1$ и $\psi_N$ нам известны~\eqref{eq:newWaveFuncBordCond}, а $\psi_2$ и $\psi_{N-1}$ --- нет.
Однако, если $N$ достаточно, то для простоты можно считать, что $\psi_2=d2, \psi_{N-1}=d2,$~где $d1, d2$ --- малые числа.

Для оценки близости $E$ к собственному значению будем вычислять разность производных волновых функций, полученных интегрированием \guillemetleftвперёд\guillemetright и \guillemetleftназад\guillemetright, в некотором внутреннем узле сетки $x_m$:

\begin{equation}
    f(E) = \frac{d\psi_>x}{dx}\bigg|_{x_m} - \frac{d\psi_<x}{dx}\bigg|_{x_m}.
    \label{eq:inner_node_deriv_diff}
\end{equation}
где
\begin{equation}
    \frac{d\psi}{dx}\bigg|_{x_m} =\frac{\psi(x_m-2h) - \psi(x_m+2h) + 8\left[ \psi(x_m + h) - \psi(x_m - h) \right]}{12h}.
    \label{eq:inner_node_deriv_diff_expl}
\end{equation}
Здесь $\psi_>, \psi_<$ - волновые функции полученные интегрированием~\guillemetleftвперёд\guillemetright~и~\guillemetleftназад\guillemetright~соответственно, $x_m$ --- узел сшивки производных.
Естественно, перед вычислением~\eqref{eq:inner_node_deriv_diff} необходимо масштабировать функции $\psi_>$ и $\psi_<$ так, чтобы $\psi_>(x_m)=\psi_<(x_m)$.

Будем увеличивать энергию с шагом $\triangle E$ до тех пор, пока величины $f^{(i)}$ на двух соседних шагах $i$ и $i - 1$ не будут иметь различные знаки.
Далее для уточнения собственного значения с наперед заданной точностью $\epsilon$ используется метод бисекции.

\newpage

\section{Программная реализация алгоритма}\label{sec:--}
В~\nameref{sec:extras} представлена программа на языке Python 3.12\cite{python},
реализованная в среде разработки PyCharm Community Edition 2024.3.1,
численного решения одномерного стационарного уравнения Шрёдингера для электрона в одномерной потенциальной яме.
Программа реализует алгоритм теории возмущений,
позволяющий находить собственные значения и соответствующие им волновые функции.
Потенциальная функция (невозмущенная система) и параметры для нее соответствуют постановке задачи из первой главы.
Энергия и длина ямы были переведены в атомные единицы Хартри (строки~\textbf{Х}--\textbf{Х}).


В строках~\textbf{Х}--\textbf{Х} реализован алгоритм пристрелки, который подробно разобран в лабораторной работе №1,
в данной программе этот алгоритм используется для реализации невозмущенной системы и вычисления её решения
(собственные значения и собственные функции оператора Гамильтона).
Собственные значения и собственные функции невозмущенной системы будут использоваться для вычисления решения возмущенной системы.
Путем компьютерного моделирования был вычислен верхний предел сумм~\eqref{eq:E_Psi_n_sum}.
П программе за верхний предел сумм отвечает~\lstinline{k_max} в строке~\textbf{X},
он равен количеству вычисленных энергий невозмущенной системы, для длинного варианта задачи достаточно было 14.


В строках~\textbf{Х}--\textbf{Х} реализована потенциальная функция возмущенной системы, для этого был создан пик, который больше~$\max(U(x))$ на отрезке $\left[ 2.5; 3.0 \right]$


В строках~\textbf{Х}--\textbf{Х} реализован оператор возмущения.


В строках~\textbf{Х}--\textbf{Х} и~\textbf{Х}--\textbf{Х} реализованы функции возвращающие энергии и волновые функции невозмущенной системы.


В строках~\textbf{Х}--\textbf{Х} и~\textbf{Х}--\textbf{Х} реализованы функции которые вычисляют матричный элемент оператора возмущения по невозмущенным системам~\eqref{eq:EPsi_nk_first}.


В строках~\textbf{Х}--\textbf{Х} реализована функция вычисляющая поправку второго порядка~\eqref{eq:E_Psi_n_sum}.


В строках~\textbf{Х}--\textbf{Х} и~\textbf{Х}--\textbf{Х} реализованы функции вычисляющие поправку первого порядка для волновой функции возмущенной системы.


В строках~\textbf{Х}--\textbf{Х} реализована функция вычисляющая первое приближение волновой функции~\eqref{eq:first_order_e_psi}


В строках~\textbf{Х}--\textbf{Х} реализована функция с одним параметром~\lstinline{root} определяющий номер состояния возмущенной системы для которого требуется вычислить энергию и волновую функцию.
В функции вычисляется энергия с учетом поправок до второго порядка включительно, и волновая функция с учетом поправок первого порядка.
Также в функции реализован вывод графиков и запись данных в файл.


В строках~\textbf{Х}--\textbf{Х} вызывается функция\lstinline{result} для основного и второго возбужденного состояний возмущенной системы.


В строках~\textbf{Х}--\textbf{Х} выводятся графики невозмущенной и возмущенной систем.

\section{Результаты численных экспериментов}\label{sec:results}

Ниже продемонстрированы результаты работы программного кода написанного на Python.

\subsection{Иллюстрация работы программы}\label{subsec:results_images}

Потенциал из постановки задачи представлен на Рис.~\ref{fig:pot_func}

\begin{figure}[h]
\centering
    \includegraphics[width=0.45\linewidth]{Potential_func_graph}
    \caption{Вероятностная плотность}\label{fig:pot_func}
\end{figure}

Для основного состояния была получена энергия $E = 0.026429$ и следующая волновая функция (Рис.~\ref{fig:norm0}) и плотность вероятности (Рис.~\ref{fig:probDens0}):

\begin{figure}[H]
    \centering
    \begin{subfigure}{0.45\textwidth}
        \centering
        \includegraphics[width=0.9\linewidth]{Condition_0_(normalized)}
        \caption{Нормализованное состояние}
        \label{fig:norm0}
    \end{subfigure}%
    \begin{subfigure}{0.45\textwidth}
        \centering
        \includegraphics[width=0.9\linewidth]{Condition_0_(Probability_density)}
        \caption{Вероятностная плотность}
        \label{fig:probDens0}
    \end{subfigure}%
\caption{Графики для основного состояния}
\end{figure}
\label{fig:cond0}

При анализе заданной потенциала было обнаружено: основным состоянием для него является состояние при котором волновая функция имеет два пересечения с осью абсцисс.

Для второго возбужденного состояния была получена энергия $E = 0.815564$ и следующая волновая функция (Рис.~\ref{fig:norm2}) и плотность вероятности (Рис.~\ref{fig:probDens2}):

\begin{figure}[H]
    \centering
    \begin{subfigure}{0.45\textwidth}
        \centering
        \includegraphics[width=0.9\linewidth]{Condition_2_(normalized)}
        \caption{Нормализованное состояние}
        \label{fig:norm2}
    \end{subfigure}%
    \begin{subfigure}{0.45\textwidth}
        \centering
        \includegraphics[width=0.9\linewidth]{Condition_2_(Probability_density)}
        \caption{Вероятностная плотность}
        \label{fig:probDens2}
    \end{subfigure}%
\caption{Графики для состояния 2}
\end{figure}
\label{fig:cond2}

На рисунке~(\ref{fig:norm2}) можно заметить что волновая функция имеет 4 пересечения с осью абсцисс.
Соответственно, согласно осцилляционной теореме, функция соответствует второму возбужденному состоянию так как есть четыре пересечения с осью абсцисс,
кроме бесконечных потенциальных стенок которые не учитываются.

\subsection{Значения искомых параметров}\label{subsec:results_values}

Ниже результаты численных экспериментов, полученных в результате работы программы выведены в таблицу:

Квантовомеханические средние $\langle p(x) \rangle$ и $\langle p(x^2) \rangle$ для основного, первого и второго возбужденного состояний:


\noindent
\begin{tabularx}{\linewidth}{|c|X|X|X|}
    \hline
    \textbf{Состояние}&\textbf{Энергия, а.е.}&\textbf{$\langle p(x) \rangle$}&\textbf{$\langle p(x^2) \rangle$} \\
    \hline
    Основное & $0.026429$ & $0.000000e+00$ & $3.070530e-01$\\
    \hline
    2-е возбужденное & $0.815564$ & $0.000000e+00$ & $2.435363e+00$\\
    \hline
\end{tabularx}

\newpage

\section{Заключение}\label{sec:zakl}

Таким образом, было получено численное решение для задачи о частице в одномерной квантовой яме с бесконечными стенками при помощи метода пристрелки.
Были получены значения энергий и волновые функции основного и второго возбужденного состояний.
Полученные волновые функции соответствуют осцилляционной теореме,
в данном варианте можно четко заметить что, по сравнению с основным состоянием, количество пересечений с осью абсцисс увеличилось на два.
Кроме того, для каждого состояния были вычисленные квантовомеханические средние~$\langle p(x) \rangle, \langle p(x^2) \rangle$.

\newpage

\appendix

\section*{Приложение}\label{sec:extras}
\input{latex_tex/extras}

\end{document}