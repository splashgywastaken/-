В~\nameref{sec:extras} программная реализация задачи выполнена на языке Python~3.12~\cite{python}, реализованная в среде разработки PyCharm Community Edition 2024.3.1, численного решения одномерного стационарного уравнения Шредингера для электрона в одномерной потенциальной яме.
Программа реализует алгоритм пристрелки, позволяющей находить собственные значения и соответствующие им волновые функции.
Потенциальная функция и параметры для нее соответствуют постановке задачи из первой главы (\ref{sec:goals}).


Код программы написан таким образом, что решение поставленной задачи выведено в отдельный файл \textbf{solve.py}.
В данном файле приводится класс решающий основную часть задачи, а также вспомогательные функции.
Рассмотрим код данного файла.

Вспомогательная функция~\lstinline[language=Python, columns=fixed]{draw_potential_graph()} отвечает за отрисовку отдельно от основного кода графика потенциальной функции.


Ниже рассматривается код класса~\lstinline[columns=fixed]{Solver} -- класса, отвечающего за решение поставленной задачи:

Задание параметров из постановки задачи происходит в конструкторе класса~\lstinline[columns=fixed]{Solver}, такие как: количество точек системы $n$, параметры потенциальной функции $L,V_0$, задаётся приближенное значение энергии $E_{min}$ и шаг $step$, также задаётся узел сшивки $r$.
Кроме того в нем же энергия и длина ямы были переведены в атомные единицы Хартри (строки 46--49).

В методе~\lstinline[columns=fixed]{u_func} (строки 64--75) реализована потенциальная функция $U(x)$ из постановки задачи.


В методе~\lstinline[columns=fixed]{find_exact_energies} класса (строки 147--158) реализован метод пристрелки соответсвующий алгоритму из главы~\ref{sec:shooting_method}.


Реализованы методы~\lstinline[columns=fixed]{energy_scan} и~\lstinline[columns=fixed]{bisection_method} в строках 136--145 и 160--172 соответственно.
Метод~\lstinline[columns=fixed]{find_exact_energies} вычисляет все значения энергий на отрезке~$\left[ E_{min}, E_{max} \right]$ с шагом~\lstinline[columns=fixed]{step},
метод~\lstinline[columns=fixed]{bisection_method} реализует метод бисекции.


В строках 108--134 реализован метод в котором с помощью метода Нумерова вычисляются волновые функции интегрированием вперед и назад, а также вычисляется разность этих функций в узле сшивки.
В строках 77--78 реализована часть уравнения Шрёдингера и в строках 81--83 реализована формула для вычисления производной в узле сшивки.


В строках 86--88 реализована метод для нормировки волновой функции.


В строках 92--97 реализовано вычисление квантовомеханических средних по формуле
$\langle P_x \rangle = \int_{a}^{b}\Psi_n(x)\hat{P}\Psi_n(x),dx = -i \hslash\int_{a}^{b}\Psi_n(x)\frac{d\Psi_n(x)}{dx},dx$.


В строках 174--212 реализован метод для вывода графиков и записи данных в файл.


В строках 215--225 реализован основной метод класса вызывается метод пристрелки и происходит вычисление всех состояний на заданном диапазоне.


\newpage