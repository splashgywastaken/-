\documentclass[a4paper,12pt]{article}


\usepackage{mathtext}
\usepackage[T2A]{fontenc}
\usepackage[utf8]{inputenc}
\usepackage[russian]{babel}
\usepackage{amsmath}
\usepackage{amsfonts}
\usepackage{amssymb}
\usepackage{graphicx}
\usepackage{nameref}
\usepackage{subcaption}
\usepackage[margin=1.5pt]{geometry}
\usepackage{float}
\usepackage{latexsym}
\usepackage{stmaryrd}
\usepackage{hyperref}
\usepackage{listings}
\usepackage{xcolor}
\usepackage{tabularx}


\geometry{left=2cm, right=2cm, top=2cm, bottom=2cm}

\begin{document}

\definecolor{codegreen}{rgb}{0,0.6,0}
\definecolor{codegray}{rgb}{0.5,0.5,0.5}
\definecolor{codepurple}{rgb}{0.58,0,0.82}
\definecolor{backcolour}{rgb}{1,1,1}

\lstdefinestyle{pythonStyle}{
    backgroundcolor=\color{backcolour},
    commentstyle=\color{codegreen},
    keywordstyle=\color{red},
    numberstyle=\tiny\color{codegray},
    stringstyle=\color{codepurple},
    basicstyle=\ttfamily\footnotesize,
    breakatwhitespace=false,
    breaklines=true,
    captionpos=b,
    keepspaces=true,
    numbers=left,
    numbersep=5pt,
    showspaces=false,
    showstringspaces=false,
    showtabs=false,
    tabsize=2
}
\lstset{style=pythonStyle}

\pagestyle{empty}

\begin{center}
\textbf{МИНОБРНАУКИ РОССИИ}\\
ФЕДЕРАЛЬНОЕ ГОСУДАРСТВЕННОЕ БЮДЖЕТНОЕ \\
ОБРАЗОВАТЕЛЬНОЕ УЧРЕЖДЕНИЕ ВЫСШЕГО ОБРАЗОВАНИЯ \\
ВОРОНЕЖСКИЙ ГОСУДАРСТВЕННЫЙ УНИВЕРСИТЕТ \\
Факультет прикладной математики, информатики и механики\\
Кафедра вычислительной математики и прикладных информационных технологий
\end{center}

\vspace{2cm}
\begin{center}
\textbf{ЛАБОРАТОРНАЯ РАБОТА №2}\\
\textbf{ЧИСЛЕННОЕ РЕШЕНИЕ СТАЦИОНАРНОГО УРАВНЕНИЯ ШРЁДИНГЕРА: ТЕОРИЯ ВОЗМУЩЕНИЙ}
\end{center}

\vspace{3cm}
\begin{flushright}
\begin{tabular}{l l}
\textbf{Направление:} & 01.04.02 \textendash{} Прикладная математика и информатика \\
\textbf{Выполнил:} & студент 11 группы 2 курса магистратуры \\
& Крутько А.С. \\
\textbf{Преподаватель:} & доктор физ.-мат. наук, профессор Тимошенко Ю.К.
\end{tabular}
\end{flushright}

\vspace{3cm}
\begin{center}
Воронеж 2024
\end{center}

\newpage
\tableofcontents
\pagestyle{plain}
\setcounter{page}{2}

\newpage
\section{Цели и задачи работы}\label{sec:goals}
\subsection{Цель работы.}\label{subsec:-.}
Целями лабораторной работы являются практическое освоение информации,
полученной при изучении курса <<Компьютерное моделирование в математической физике>> по теме <<Численное решение стационарного уравнения Шрёдингера>>,
а также развитие алгоритмического мышления и приобретение опыта использования знаний и навыков по математике,
численным методам и программированию для решения прикладных задач физико-технического характера.

\subsection{Задачи работы:}\label{subsec:-:}

\textbf{Проблема:} электрон находится в одномерной потенциальной яме с бесконечными стенками $U(x)$:
\[
v(x) =
    \begin{cases}
        J_2(x), & x \in (-L, L), \\
        \infty, & x \notin (-L, L),
    \end{cases}
\]
Где $U(x) = v(x)*V_0$, $V_0 = 25$ \text{эВ}, $L = 3$ \AA, $J_n(x)$ -- функция Бесселя, $n = 2$.

\begin{enumerate}
    \item Используя метод возмущений, найти энергию, нормированную волновую функцию для основного и 2-го возбужденного состояний и плотности вероятности.
          Энергию вычислять с учетом поправок до второго порядка включительно, а волновую функцию с учетом поправок первого порядка.
          Возмущенную систему смоделировать, создав в потенциальной функции $U(x)$ пик произвольной формы.
          Привести как числовые значения энергий, так и построить графики волновых функций и плотностей вероятности.
    \item Вычислить для этих состояний квантовомеханические средние $\langle p(x) \rangle$ и $ \langle p(x^2) \rangle $.
    \item Сравнить результаты с данными, полученными методом пристрелки.
\end{enumerate}

\newpage
\section{Одномерное стационарное уравнение Шрёдингера. Математический формализм. Общие свойства решений}\label{sec:-}
Одномерное стационарное уравнение Шрёдингера~\cite{tim_shrod}:
\begin{equation}
    \hat{H}\psi(x) = E\psi(x),
    \label{eq:oneDimShrodingerEq}
\end{equation}
где $\hat{H}$ \textendash{} оператор Гамильтона, $E$ \textendash{} собственные значения энергии, $\psi(x)$ \textendash{} волновая функция.

С математической точки зрения оно представляет собой задачу определения собственных значений $E$ и собственных функций $\psi$ оператора Гамильтона $\hat{H}$.
Для частицы с массой $m$, находящейся в потенциальном поле $U(x)$, оператор Гамильтона имеет вид
\begin{equation}
    \hat{H} = \hat{T}+ U(x),
    \label{eq:equation}
\end{equation}
где оператор кинетической энергии
\begin{equation}
    \hat{T} =-\frac{\hslash^2}{2m}\frac{d^2}{dx^2},
    \label{eq:equation2}
\end{equation}
а $\hslash$ --- постоянная Планка.
Собственное значение оператора Гамильтона имеет смысл энергии соответствующей изолированной квантовой системы.
Собственные функции называются волновыми функциями.
Волновая функция однозначна и непрерывна во всём пространстве.
Непрерывность волновой функции и её первой производной сохраняется и при обращении $U(x)$ в $\infty$ некоторой области пространства.
В такую область частица вообще не может проникнуть, то есть в этой области, а также на её границе $\psi(x)=0$.

Оценим нижнюю границу энергетического спектра.
Пусть минимальное значение потенциальной функции равно $U_{\min}$.
Очевидно, что $\langle T \rangle \geq 0$ и $\langle U \rangle \geq U_{\min}$.
Потому из уравнения~\eqref{eq:oneDimShrodingerEq} следует, что:
\begin{equation}
    E =\langle H\rangle = \int_{-\infty}^{+\infty} \psi^\star(x)\hat{H}\psi(x) \,dx =\langle T\rangle +\langle U\rangle > U_{\min}.
\label{eq:e_h_integral}
\end{equation}
то есть, энергии всех состояний > $U_{min}$.

Особый практический интерес представляет случай, когда
\begin{equation}
    \lim_{x\to\infty} U(x) = 0.
\label{eq:limit_pot_inf}
\end{equation}

Потенциал такого типа называется также потенциальной ямой.
Для данной $U(x)$ свойства решений уравнения Шрёдингера зависят от знака собственного значения $E$.
Если $E < 0$.
Частица с отрицательной энергией совершает финитное движение.
Оператор Гамильтона имеет дискретный спектр, то есть собственные значения и соответствующие собственные функции можно снабдить номерами.
При~$E < 0$ уравнение~\eqref{eq:oneDimShrodingerEq} приобретает вид\cite{tim_shrod}:

\begin{equation}
    \hat{H}\psi_k(x) = E_k\psi_k(x).
    \label{eq:shrodinger_eq_e_less_0}
\end{equation}

Квантовое состояние, обладающее наименьшей энергией, называется основным.
Остальные состояния называют возбужденными состояниями.
В силу линейности стационарного уравнения Шрёдингера, волновые функции математически определены с точностью до постоянного множителя.
Однако, из физических соображений, волновые функции должны быть нормированы следующим образом:

\begin{equation}
    \int_{-\infty}^{+\infty} |\psi_k(x)|^2, dx = 1.
    \label{eq:wave_func_normalization}
\end{equation}

В дальнейшем будет рассматриваться только дискретный спектр.
При этом необходимо пользоваться \textbf{осцилляционной теоремой}.

\textbf{Осцилляционная теорема.}
Упорядочим собственные значения оператора Гамильтона в порядке возрастания, нумеруя энергию основного состояния индексом "0": $E_0$, $E_1$, $E_2$ \dots, $E_k$,\dots.
Тогда волновая функция $\psi_k(x)$ будет иметь $k$ узлов (то есть, пересечений с осью абсцисс).
Исключения: области, в которых потенциальная функция бесконечна.

\newpage

\section{Теория возмущений. Алгоритм}\label{sec:solve_method}
К числу приближенных методов вычисления собственных значений и собственных функций оператора Гамильтона относится метод стационарных возмущений Релея-Шрёдингера\cite{tim_pertrubations}, который мы далее будем просто называть «методом» или «теорией возмущений».


В рамках этого теоретического подхода предполагается, что оператор Гамильтона, чьи собственные значения и собственные функции требуется определить, может быть представлен в виде:
\begin{equation}
    \label{eq:gamilton_op}
    \hat{H} = \hat{H}^0 + \hat{V}
\end{equation}
где $\hat{H}^0$ -- гамильтониан идеализированной задачи, решение которой можно найти либо аналитически, либо относительно простым численным путем;~\hat{V} -- называется оператором возмущения или просто возмущением.


Оператором возмущения может быть либо часть гамильтониана, которая не учитывалась в идеализированной задаче, либо потенциальная энергия, связанная с наличием внешнего воздействия.


Идеализированную систему, которую описывает гамильтониан~$\hat{H}^0$, называют «невозмущенной» системой, а систему с гамильтонианом~$\hat{H}$ -- «возмущенной» системой.
В рамках теории возмущений удаётся получить формулы, определяющие энергии и волновые функции стационарных состояний через известные значения энергий~$E_n^{(0)}$ и волновых функций~$\Psi_n{(0)}$ невозмущенной системы.


Стационарные уравнения Шрёдингера для невозмущенной ив возмущенной систем~\cite{tim_pertrubations} имеют вид:
\begin{equation}
\label{eq:stationary_eq_shrod0}
\hat{H}^{(0)}\Psi_n^{(0)}(x)=E_n^{(0)}\Psi_n^{(0)};
\end{equation}

\begin{equation}
\label{eq:stationary_eq_shrod}
\hat{H}\Psi_n(x)=E_n\Psi_n(x).
\end{equation}


В теории возмущений решения уравнения~\eqref{eq:stationary_eq_shrod} ищутся в виде рядов:


\begin{equation}
\label{eq:E_Psi_n_sum}
\begin{split}
    &E_n=E_n^{(0)} + E_n^{(1)} + E_n^{(2)} +\dots =\sum_{k=0}^{\infty}E_n^{(k)},\\
    &\Psi_n(x)=\Psi_n^{(0)}(x) + \Psi_n^{(1)}(x) + \Psi_n^{(2)}(x) +\dots =\sum_{k=0}^{\infty}\Psi_n^{(k)}(x),
\end{split}
\end{equation}
где~$E_n^{k}$, $\Psi_n^{(k)}$ -- величины $k$-го порядка малости по возмущению $\hat{V}$, называемые $k$-ми поправками или поправками $k$-го порядка.
Первые слагаемые рядов~\eqref{eq:E_Psi_n_sum} определяются следующими формулами:
\begin{equation}
\begin{split}
    \label{eq:EPsi_nk_first}
    &E_n^{(1)}=V_{nn}, \\
    &E_n^{(2)}=\sum_m^{\prime}{\frac{|V_{mn}|^2}{E_n^{(0)}-E_m^{(0)})}}, \\
    &\Psi_n^{(0)}(x)=\sum_m^{\prime}{\frac{V_{mn}}{E_n^{(0)}-E_m^{(0)}}}\Psi_m^{0}(x),
\end{split}
\end{equation}
где
\begin{equation}
    \label{eq:v_mn}
    V_{mn}\equiv \langle{m|V|n}\rangle=\int_{\Omega}{\Psi_m^{(0)\star}(x)\hat{V}\Psi_n^{(0)}(x)dx.}
\end{equation}


Штрих над знаком суммы означает пропуск слагаемого с $m=n: \sum_m^{\prime}\equiv \sum_{m\neq n}$.
Очевидно, что ряды в~\eqref{eq:E_Psi_n_sum} сходятся, если выполняется равенство

\begin{equation}
    \label{eq:eq_for_conj}
    |V_{mn}|\ll|E_n^{(0)}-E_m^{(0)}|.
\end{equation}


Во многих случаях для решения задачи достаточно ограничиться вычислением энергии с учетом поправок до второго порядка включительно волновой функции с учетом поправок первого порядка\cite{tim_pertrubations}:

\begin{equation}
\label{eq:first_order_e_psi}
\begin{split}
    &E_n = E_n^{(0)}+V_{nn}+\sum_m^{\prime}{\frac{|V_{mn}|^2}{E_n^{(0)}-E_m^{(0)}}}, \\
    &\Psi_n(x)=\Psi_n^{(0)}(x)+\sum_m^{\prime}{\frac{V_{mn}}{E_n^{(0)}-E_m^{(0)}}\Psi_m^{(0)}(x)}.
\end{split}
\end{equation}


При вычислении второй поправки к энергии и первой поправки к волновой функции основного состояния возмущенной системы бесконечные суммы в~\eqref{eq:E_Psi_n_sum} заменяются конечными.
Для оценки корректного численного значения надо выполнить несколько расчетов соответствующей суммы для монотонно возрастающих величин.
Если увеличение верхнего предела суммы, начиная с некоторого значения, не приводит к заметным изменениям суммы, то задача оценки значения верхнего значения суммы решена.

\newpage

\section{Программная реализация алгоритма}\label{sec:--}
В~\nameref{sec:extras} представлена программа на языке Python 3.12\cite{python},
реализованная в среде разработки PyCharm Community Edition 2024.3.1,
численного решения одномерного стационарного уравнения Шрёдингера для электрона в одномерной потенциальной яме.
Программа реализует алгоритм теории возмущений,
позволяющий находить собственные значения и соответствующие им волновые функции.
Потенциальная функция (невозмущенная система) и параметры для нее соответствуют постановке задачи из первой главы.
Энергия и длина ямы были переведены в атомные единицы Хартри (строки~\textbf{Х}--\textbf{Х}).


В строках~\textbf{Х}--\textbf{Х} реализован алгоритм пристрелки, который подробно разобран в лабораторной работе №1,
в данной программе этот алгоритм используется для реализации невозмущенной системы и вычисления её решения
(собственные значения и собственные функции оператора Гамильтона).
Собственные значения и собственные функции невозмущенной системы будут использоваться для вычисления решения возмущенной системы.
Путем компьютерного моделирования был вычислен верхний предел сумм~\eqref{eq:E_Psi_n_sum}.
П программе за верхний предел сумм отвечает~\lstinline{k_max} в строке~\textbf{X},
он равен количеству вычисленных энергий невозмущенной системы, для длинного варианта задачи достаточно было 14.


В строках~\textbf{Х}--\textbf{Х} реализована потенциальная функция возмущенной системы, для этого был создан пик, который больше~$\max(U(x))$ на отрезке $\left[ 2.5; 3.0 \right]$


В строках~\textbf{Х}--\textbf{Х} реализован оператор возмущения.


В строках~\textbf{Х}--\textbf{Х} и~\textbf{Х}--\textbf{Х} реализованы функции возвращающие энергии и волновые функции невозмущенной системы.


В строках~\textbf{Х}--\textbf{Х} и~\textbf{Х}--\textbf{Х} реализованы функции которые вычисляют матричный элемент оператора возмущения по невозмущенным системам~\eqref{eq:EPsi_nk_first}.


В строках~\textbf{Х}--\textbf{Х} реализована функция вычисляющая поправку второго порядка~\eqref{eq:E_Psi_n_sum}.


В строках~\textbf{Х}--\textbf{Х} и~\textbf{Х}--\textbf{Х} реализованы функции вычисляющие поправку первого порядка для волновой функции возмущенной системы.


В строках~\textbf{Х}--\textbf{Х} реализована функция вычисляющая первое приближение волновой функции~\eqref{eq:first_order_e_psi}


В строках~\textbf{Х}--\textbf{Х} реализована функция с одним параметром~\lstinline{root} определяющий номер состояния возмущенной системы для которого требуется вычислить энергию и волновую функцию.
В функции вычисляется энергия с учетом поправок до второго порядка включительно, и волновая функция с учетом поправок первого порядка.
Также в функции реализован вывод графиков и запись данных в файл.


В строках~\textbf{Х}--\textbf{Х} вызывается функция\lstinline{result} для основного и второго возбужденного состояний возмущенной системы.


В строках~\textbf{Х}--\textbf{Х} выводятся графики невозмущенной и возмущенной систем.

\newpage

\section{Результаты численных экспериментов}\label{sec:results}

Ниже продемонстрированы результаты работы программного кода написанного на Python.

\subsection{Иллюстрация работы программы}\label{subsec:results_images}

Потенциал из постановки задачи представлен на Рис.~\ref{fig:pot_func}

%\begin{figure}[h]
%\centering
%    \includegraphics[width=0.45\linewidth]{}
%    \caption{Вероятностная плотность}\label{fig:pot_func}
%\end{figure}

%\begin{figure}[H]
%    \centering
%    \begin{subfigure}{0.45\textwidth}
%        \centering
%        \includegraphics[width=0.9\linewidth]{Condition_0_(normalized)}
%        \caption{Нормализованное состояние}
%        \label{fig:norm0}
%    \end{subfigure}%
%    \begin{subfigure}{0.45\textwidth}
%        \centering
%        \includegraphics[width=0.9\linewidth]{Condition_0_(Probability_density)}
%        \caption{Вероятностная плотность}
%        \label{fig:probDens0}
%    \end{subfigure}%
%\caption{Графики для состояния 0}
%\end{figure}
%\label{fig:cond0}



%\begin{figure}[H]
%    \centering
%    \begin{subfigure}{0.45\textwidth}
%        \centering
%        \includegraphics[width=0.9\linewidth]{Condition_2_(normalized)}
%        \caption{Нормализованное состояние}
%        \label{fig:norm2}
%    \end{subfigure}%
%    \begin{subfigure}{0.45\textwidth}
%        \centering
%        \includegraphics[width=0.9\linewidth]{Condition_2_(Probability_density)}
%        \caption{Вероятностная плотность}
%        \label{fig:probDens2}
%    \end{subfigure}%
%\caption{Графики для состояния 1}
%\end{figure}
%\label{fig:cond2}

\subsection{Значения искомых параметров}\label{subsec:results_values}

Ниже результаты численных экспериментов, полученных в результате работы программы выведены в таблицу:

Квантовомеханические средние $\langle p(x) \rangle$ и $\langle p(x^2) \rangle$ для основного, первого и второго возбужденного состояний:

\noindent
\begin{tabularx}{\linewidth}{|c|X|X|X|}
    \hline
    \textbf{Состояние}&\textbf{Энергия, а.е.}&\textbf{$\langle p(x) \rangle$}&\textbf{$\langle p(x^2) \rangle$} \\
    \hline
    Основное & $0.026451$ & $0.000000e+00$ & $3.072237e-01$\\
    \hline
    1-е возбужденное & $0.498856$ & $0.000000e+00$ & $1.006510e+00$\\
    \hline
    2-е возбужденное & $0.828367$ & $0.000000e+00$ & $2.410640e+00$\\
    \hline
\end{tabularx}

\newpage

\section{Заключение}\label{sec:zakl}

Таким образом, было получено численное решение для задачи о частице в одномерной квантовой яме с бесконечными стенками при помощи метода пристрелки.
Были получены значения энергий и волновые функции основного и второго возбужденного состояний.
Полученные волновые функции соответствуют осцилляционной теореме.
Кроме того, для каждого состояния были вычисленные квантовомеханические средние~$\langle p(x) \rangle, \langle p(x^2) \rangle$.

\newpage

\appendix

\section*{Приложение}\label{sec:extras}
\input{extras}


\newpage
\begin{thebibliography}{9}
\bibitem{tim_shrod} Тимошенко Ю.К. \textit{Численное решение стационарного уравнения Шрёдингера.} Воронеж, 2019.
\bibitem{python} Доля П.Г. \textit{Введение в научный Python} Харьков: ХНУ, 2016. 265 с.
\bibitem{tim_pertrubations} Тимошенко Ю.К. \textit{Численное решение стационарного уравнения Шрёдингера: теория возмущений. Учебное пособие} Воронеж: Научная книга, 2019. 14с.
\end{thebibliography}


\end{document}