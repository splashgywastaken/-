К числу приближенных методов вычисления собственных значений и собственных функций оператора Гамильтона относится метод стационарных возмущений Релея-Шрёдингера\cite{tim_pertrubations}, который мы далее будем просто называть «методом» или «теорией возмущений».


В рамках этого теоретического подхода предполагается, что оператор Гамильтона, чьи собственные значения и собственные функции требуется определить, может быть представлен в виде:
\begin{equation}
    \label{eq:gamilton_op}
    \hat{H} = \hat{H}^0 + \hat{V}
\end{equation}
где $\hat{H}^0$ -- гамильтониан идеализированной задачи, решение которой можно найти либо аналитически, либо относительно простым численным путем;~\hat{V} -- называется оператором возмущения или просто возмущением.


Оператором возмущения может быть либо часть гамильтониана, которая не учитывалась в идеализированной задаче, либо потенциальная энергия, связанная с наличием внешнего воздействия.


Идеализированную систему, которую описывает гамильтониан~$\hat{H}^0$, называют «невозмущенной» системой, а систему с гамильтонианом~$\hat{H}$ -- «возмущенной» системой.
В рамках теории возмущений удаётся получить формулы, определяющие энергии и волновые функции стационарных состояний через известные значения энергий~$E_n^{(0)}$ и волновых функций~$\Psi_n{(0)}$ невозмущенной системы.


Стационарные уравнения Шрёдингера для невозмущенной ив возмущенной систем~\cite{tim_pertrubations} имеют вид:
\begin{equation}
\label{eq:stationary_eq_shrod0}
\hat{H}^{(0)}\Psi_n^{(0)}(x)=E_n^{(0)}\Psi_n^{(0)};
\end{equation}

\begin{equation}
\label{eq:stationary_eq_shrod}
\hat{H}\Psi_n(x)=E_n\Psi_n(x).
\end{equation}


В теории возмущений решения уравнения~\eqref{eq:stationary_eq_shrod} ищутся в виде рядов:


\begin{equation}
\label{eq:E_Psi_n_sum}
\begin{split}
    &E_n=E_n^{(0)} + E_n^{(1)} + E_n^{(2)} +\dots =\sum_{k=0}^{\infty}E_n^{(k)},\\
    &\Psi_n(x)=\Psi_n^{(0)}(x) + \Psi_n^{(1)}(x) + \Psi_n^{(2)}(x) +\dots =\sum_{k=0}^{\infty}\Psi_n^{(k)}(x),
\end{split}
\end{equation}
где~$E_n^{k}$, $\Psi_n^{(k)}$ -- величины $k$-го порядка малости по возмущению $\hat{V}$, называемые $k$-ми поправками или поправками $k$-го порядка.
Первые слагаемые рядов~\eqref{eq:E_Psi_n_sum} определяются следующими формулами:
\begin{equation}
\begin{split}
    \label{eq:EPsi_nk_first}
    &E_n^{(1)}=V_{nn}, \\
    &E_n^{(2)}=\sum_m^{\prime}{\frac{|V_{mn}|^2}{E_n^{(0)}-E_m^{(0)})}}, \\
    &\Psi_n^{(0)}(x)=\sum_m^{\prime}{\frac{V_{mn}}{E_n^{(0)}-E_m^{(0)}}}\Psi_m^{0}(x),
\end{split}
\end{equation}
где
\begin{equation}
    \label{eq:v_mn}
    V_{mn}\equiv \langle{m|V|n}\rangle=\int_{\Omega}{\Psi_m^{(0)\star}(x)\hat{V}\Psi_n^{(0)}(x)dx.}
\end{equation}


Штрих над знаком суммы означает пропуск слагаемого с $m=n: \sum_m^{\prime}\equiv \sum_{m\neq n}$.
Очевидно, что ряды в~\eqref{eq:E_Psi_n_sum} сходятся, если выполняется равенство

\begin{equation}
    \label{eq:eq_for_conj}
    |V_{mn}|\ll|E_n^{(0)}-E_m^{(0)}|.
\end{equation}


Во многих случаях для решения задачи достаточно ограничиться вычислением энергии с учетом поправок до второго порядка включительно волновой функции с учетом поправок первого порядка\cite{tim_pertrubations}:

\begin{equation}
\label{eq:first_order_e_psi}
\begin{split}
    &E_n = E_n^{(0)}+V_{nn}+\sum_m^{\prime}{\frac{|V_{mn}|^2}{E_n^{(0)}-E_m^{(0)}}}, \\
    &\Psi_n(x)=\Psi_n^{(0)}(x)+\sum_m^{\prime}{\frac{V_{mn}}{E_n^{(0)}-E_m^{(0)}}\Psi_m^{(0)}(x)}.
\end{split}
\end{equation}


При вычислении второй поправки к энергии и первой поправки к волновой функции основного состояния возмущенной системы бесконечные суммы в~\eqref{eq:E_Psi_n_sum} заменяются конечными.
Для оценки корректного численного значения надо выполнить несколько расчетов соответствующей суммы для монотонно возрастающих величин.
Если увеличение верхнего предела суммы, начиная с некоторого значения, не приводит к заметным изменениям суммы, то задача оценки значения верхнего значения суммы решена.