В~\nameref{sec:extras} представлена программа на языке Python 3.12\cite{python},
реализованная в среде разработки PyCharm Community Edition 2024.3.1,
численного решения одномерного стационарного уравнения Шрёдингера для электрона в одномерной потенциальной яме.
Программа реализует алгоритм теории возмущений,
позволяющий находить собственные значения и соответствующие им волновые функции.
Потенциальная функция (невозмущенная система) и параметры для нее соответствуют постановке задачи из первой главы.
Энергия и длина ямы были переведены в атомные единицы Хартри (строки 124--127).


В строках 6--140 реализован алгоритм пристрелки, который подробно разобран в лабораторной работе №1,
в данной программе этот алгоритм используется для реализации невозмущенной системы и вычисления её решения
(собственные значения и собственные функции оператора Гамильтона).
Собственные значения и собственные функции невозмущенной системы будут использоваться для вычисления решения возмущенной системы.
Путем компьютерного моделирования был вычислен верхний предел сумм~\eqref{eq:E_Psi_n_sum}.
П программе за верхний предел сумм отвечает~\lstinline{k_max} в строке 233,
он равен количеству вычисленных энергий невозмущенной системы, для длинного варианта задачи достаточно было 11.


В строках 154--161 реализована потенциальная функция возмущенной системы, для этого был создан пик высотой 1, который больше~$\max(U(x))$ на отрезке $\left[ -0.545; 0.545 \right]$


В строках 164--165 реализован оператор возмущения.


В строках 168--172 и 175--178 реализованы функции возвращающие энергии и волновые функции невозмущенной системы.


В строках 181--184 и 187--190 реализованы функции которые вычисляют матричный элемент оператора возмущения по невозмущенным системам~\eqref{eq:EPsi_nk_first}.


В строках 193--206 реализована функция вычисляющая поправку второго порядка~\eqref{eq:E_Psi_n_sum}.


В строках 209--215 и 218--225 реализованы функции вычисляющие поправку первого порядка для волновой функции возмущенной системы.


В строках 228--229 реализована функция вычисляющая первое приближение волновой функции~\eqref{eq:first_order_e_psi}


В строках 232--325 реализована функция с одним параметром~\lstinline{root} определяющий номер состояния возмущенной системы для которого требуется вычислить энергию и волновую функцию.
В функции вычисляется энергия с учетом поправок до второго порядка включительно, и волновая функция с учетом поправок первого порядка.
Также в функции реализован вывод графиков и запись данных в файл.


В строках 327--328 вызывается функция\lstinline{result} для основного и второго возбужденного состояний возмущенной системы.


В строках 330--344 выводятся графики невозмущенной и возмущенной систем.